% +--------------------------------------------------------------------+
% | Copyright Page
% +--------------------------------------------------------------------+

\newpage

\thispagestyle{empty}

\section*{Abstract}
Diabetic retinopathy (DR) is one of the main complications of diabetes and the leading cause of new cases of blindness. Early detection is fundamental for a good prognosis, but diagnosis is a hard, expensive and time-consuming process. The need of automating methods for DR grading was recognized time ago, but most approaches in literature require vast computational power and have not been designed with interpretability in mind.

We show that a model based in a convolutional neural network can achieve excellent performance at DR grading, comparing favorably to much larger models and achieving state-of-the-art results. Using transfer learning, we reduce to the minimum the computational requirements: the model can be trained in a few hours on domestic hardware.

We use the hidden representation learned by the model to identify images diagnostically similar to a given one and explore the possibility of using this model in a clinical setting in a series of test carried on collaboration with a professional ophthalmologist. According to the criterion of the professional, the tool correctly identifies similar images and is a helpful assistance during diagnosis.

Furthermore, we implement several interpretability tools to understand how the model makes predictions, address important concerns for clinical application (as calibration) and compare our approach to an alternative one using Vision Transformers under strict computational requirements.

\section*{Keywords}
Deep learning, convolutional neural networks, classification, computer vision, diabetic retinopathy, Vision Transformers, transfer learning, model calibration, model interpretability, medical image analysis.


