% +--------------------------------------------------------------------+
% | Copyright Page
% +--------------------------------------------------------------------+

\newpage

\thispagestyle{empty}

\section*{Resumen}
La retinopatía diabética es una de las principales complicaciones de la diabetes y la principal causa de nuevos casos de ceguera. Detectar esta enfermedad en sus primeras etapas es clave para un buen pronóstico, pero el diagnóstico es un proceso difícil, caro y que requiere tiempo. Para solventar estos inconvenientes se han desarrollado métodos que automatizan el diagnóstico, pero la mayor parte de ellos requieren grandes recursos computacionales y no son interpretables.

En este trabajo se demuestra que un modelo basado en redes convolucionales puede diagnosticar con gran precisión la retinopatía diabética a partir de  retinografías. El modelo, entrenado en pocas horas utilizando aprendizaje transferido y con requisitos computacionales mínimos, obtiene resultados punteros, superiores a los de modelos mucho más complejos.

Como ejemplo de la aplicabilidad de este modelo, hemos utilizado la representación interna aprendida por la red neuronal para identificar imágenes con características diagnósticas similares a una imagen base. Una serie de pruebas con una oftalmóloga profesional, asientan la posibilidad de utilizar el modelo en la práctica clínica: la profesional consideró que la herramienta era capaz de identificar imágenes similares consistentemente y resultaba de ayuda para realizar el diagnóstico.

También implementamos diversas técnicas de interpretabilidad para comprender el funcionamiento del modelo y abordamos posibles obstáculos para el uso práctico del modelo (como la calibración). Complementariamente, realizamos una comparación entre nuestro modelo y una aproximación basada en Vision Transformers. 

\section*{Palabras clave}
Aprendizaje profundo, redes neuronales convolucionales, clasificación, visión artificial, retinopatía diabética, Vision Transformers, aprendizaje transferido, calibración del modelo, interpretabilidad, análisis de imágenes médicas.
